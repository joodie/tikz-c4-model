% Colours

\definecolor{c4personcolor}{HTML}{08427B}
\definecolor{c4systemcolor}{HTML}{1178bd}
\definecolor{c4existingsystemcolor}{HTML}{999999}
\definecolor{c4containercolor}{HTML}{3c7fc0}
\definecolor{c4componentcolor}{HTML}{85bbf0}

% A C4 node contains 3 text parts: name, technology and description

\newbox\pgfnodepartnamebox
\newbox\pgfnodeparttechnologybox
\newbox\pgfnodepartdescriptionbox

% A C4 node is constructed vertically as follows:
%
% +-----------------
% | head decoration      - horizontal line / head / browser bar...
% | inner ysep
% | nodepartname (centerpoint = origin)
% | inner ysep
% | line width           \
% | inner ysep           |
% | nodeparttechnology   | - if not blank
% | inner ysep           /
% | line width           \
% | inner ysep           |
% | nodepartdescription  | - if not blank
% | inner ysep           /
% | foot decoration      - just a line
% +---------------
%
%
% north y is at 0.5\ht\nodepartnamebox + 0.5\dpnodepartnamebox + inner
% ysep
%
% south y is at 0.5\ht\nodepartnamebox + 0.5\dpnodepartnamebox + 4
% inner yspep + technology + description + 2 line width
%
%
% when a minimum height is set, this means that center point
% is at minimum height - (headheight + footheight + 2 ysep) / 2
%
% For the basic rectangle shapes, foot and head decorations are just
% lines
%
% The mininum height 

\pgfdeclareshape{c4 rectangle}{
  \nodeparts{name,technology,description}%

  % Nodes are centered on their label (text part)
  % Text parts are horizontally centered

  \savedanchor\northeast{%
    % Calculate x
    %
    % set x to the width of the widest box
    \pgf@x=\the\wd\pgfnodepartnamebox%
    \pgf@xc=\the\wd\pgfnodeparttechnologybox%
    \ifdim\pgf@x<\pgf@xc%
    \pgf@x=\pgf@xc%
    \fi%
    \pgf@xc=\the\wd\pgfnodepartdescriptionbox%
    \ifdim\pgf@x<\pgf@xc%
    \pgf@x=\pgf@xc%
    \fi%
    % save widest width in @xb
    \pgf@xb=\pgf@x%
    %
    % append margins
    \pgfmathsetlength\pgf@xc{\pgfkeysvalueof{/pgf/inner xsep}}%
    \advance\pgf@x by 2\pgf@xc%
    % check if not smaller than minimum width
    \pgfmathsetlength\pgf@xc{\pgfkeysvalueof{/pgf/minimum width}}%
    \ifdim\pgf@x<\pgf@xc%
    % yes, too small. Enlarge...
    \pgf@x=\pgf@xc%
    \fi%
    %
    \pgf@x=.5\pgf@x%
    % Calculate y
    %
    \pgfmathsetlength\pgf@y{\pgfkeysvalueof{/pgf/inner ysep}}%
    \advance\pgf@y by .5\ht\pgfnodepartnamebox%
  }%

  \savedanchor\southwest{%
    % Calculate x
    %
    % set x to the width of the widest box
    \pgf@x=\the\wd\pgfnodepartnamebox%
    \pgf@xc=\the\wd\pgfnodeparttechnologybox%
    \ifdim\pgf@x<\pgf@xc%
    \pgf@x=\pgf@xc%
    \fi%
    \pgf@xc=\the\wd\pgfnodepartdescriptionbox%
    \ifdim\pgf@x<\pgf@xc%
    \pgf@x=\pgf@xc%
    \fi%
    % store widest width in @xb
    \pgf@xb=\pgf@x%
    %
    % append margins
    \pgfmathsetlength\pgf@xc{\pgfkeysvalueof{/pgf/inner xsep}}%
    \advance\pgf@x by 2\pgf@xc%
    % check if not smaller than minimum width
    \pgfmathsetlength\pgf@xc{\pgfkeysvalueof{/pgf/minimum width}}%
    \ifdim\pgf@x<\pgf@xc%
    % yes, too small. Enlarge...
    \pgf@x=\pgf@xc%
    \fi%
    \pgf@x=-.5\pgf@x%
    % Calculate y
    %
    % First, is total height < minimum height?
    %
    \pgfmathsetlength{\pgf@y}{\pgfkeysvalueof{/pgf/inner ysep}}%
    % TODO: check if technology or description is empty
    \pgf@y=6\pgf@y%
    \advance\pgf@y by \ht\pgfnodepartnamebox%
    \advance\pgf@y by \dp\pgfnodepartnamebox%
    \advance\pgf@y by \ht\pgfnodeparttechnologybox%
    \advance\pgf@y by \dp\pgfnodeparttechnologybox%
    \advance\pgf@y by \ht\pgfnodepartdescriptionbox%
    \advance\pgf@y by \dp\pgfnodepartdescriptionbox%
    \advance\pgf@y by 2\pgflinewidth%
    %
    \pgfmathsetlength\pgf@yb{\pgfkeysvalueof{/pgf/minimum height}}%
    \ifdim\pgf@y<\pgf@yb%
      % yes, too small. Enlarge...
      \pgf@y=\pgf@yb%
    \fi%
    % south y is full height minus top .. halfway nodepartnamebox
    \pgf@y=-\pgf@y
    \pgfmathsetlength\pgf@yb{\pgfkeysvalueof{/pgf/inner ysep}}%
    \advance\pgf@y by \pgf@yb
    \advance\pgf@y by .5\ht\pgfnodepartnamebox%
  }%

  \inheritanchor[from=rectangle]{center}
  \inheritanchor[from=rectangle]{north}
  \inheritanchor[from=rectangle]{south}
  \inheritanchor[from=rectangle]{west}
  \inheritanchor[from=rectangle]{east}

  % TODO: set or inherit anchorborder

  % Each of the node parts will be placed on an anchor (not a *saved*
  % anchor!) named after the part

  \savedanchor\namepoint{%
    \pgf@x=-.5\wd\pgfnodepartnamebox%
    \pgf@y=-.5\ht\pgfnodepartnamebox%
    \advance\pgf@y by-.5\dp\pgfnodepartnamebox%
  }

  \anchor{name}{\namepoint}%

  \savedanchor\technologypoint{%
    \pgf@x=-.5\wd\pgfnodeparttechnologybox%
    %
    \pgfmathsetlength{\pgf@y}{\pgfkeysvalueof{/pgf/inner ysep}}%
    \pgf@y=-2\pgf@y%
    \advance\pgf@y by-\pgflinewidth%
    \advance\pgf@y by-.5\ht\pgfnodepartnamebox%
    \advance\pgf@y by-.5\dp\pgfnodepartnamebox%
    \advance\pgf@y by-\ht\pgfnodeparttechnologybox%
  }%
  \anchor{technology}{\technologypoint}%

  \savedanchor\descriptionpoint{%
    \pgf@x=-.5\wd\pgfnodepartdescriptionbox%
    %
    \pgfmathsetlength{\pgf@y}{\pgfkeysvalueof{/pgf/inner ysep}}%
    \pgf@y=-4\pgf@y%
    \advance\pgf@y by-\ht\pgfnodeparttechnologybox%
    \advance\pgf@y by-\dp\pgfnodeparttechnologybox%
    \advance\pgf@y by-2\pgflinewidth%
    \advance\pgf@y by-.5\ht\pgfnodepartnamebox%
    \advance\pgf@y by-.5\dp\pgfnodepartnamebox%
    \advance\pgf@y by-\ht\pgfnodepartdescriptionbox%
  }%
  \anchor{description}{\descriptionpoint}%

  % ... and possibly more
  \backgroundpath{% this is new
    % store lower right in xa/ya and upper right in xb/yb
    \southwest \pgf@xa=\pgf@x \pgf@ya=\pgf@y
    \northeast \pgf@xb=\pgf@x \pgf@yb=\pgf@y
    % construct main path
    \pgfpathmoveto{\pgfpoint{\pgf@xa}{\pgf@ya}}
    \pgfpathlineto{\pgfpoint{\pgf@xa}{\pgf@yb}}
    \pgfpathlineto{\pgfpoint{\pgf@xb}{\pgf@yb}}
    \pgfpathlineto{\pgfpoint{\pgf@xb}{\pgf@ya}}
    \pgfpathclose

    \namepoint
    \pgfpathmoveto{\namepoint}
    
  }
}

\pgfdeclareshape{c4 person}{
  \inheritsavedanchors[from=rectangle] % this is nearly a rectangle
  \inheritanchorborder[from=rectangle]
  \inheritanchor[from=rectangle]{center}
  \inheritanchor[from=rectangle]{north}
  \inheritanchor[from=rectangle]{south}
  \inheritanchor[from=rectangle]{west}
  \inheritanchor[from=rectangle]{east}
  % ... and possibly more
  \backgroundpath{% this is new
    % store lower left in xa/ya and upper right in xb/yb
    \southwest \pgf@xa=\pgf@x \pgf@ya=\pgf@y
    \northeast \pgf@xb=\pgf@x \pgf@yb=\pgf@y
    % compute corner of ``flipped page''

    \pgf@xc=.5\wd\pgfnodepartnamebox%
    \pgf@yc=\pgf@yb \advance\pgf@yc by 5mm

    \pgfpathmoveto{\pgfpoint{\pgf@xa}{\pgf@ya}}
    \pgfpathlineto{\pgfpoint{\pgf@xa}{\pgf@yb}}
    \pgfpathlineto{\pgfpoint{\pgf@xb}{\pgf@yb}}
    \pgfpathlineto{\pgfpoint{\pgf@xb}{\pgf@ya}}
    \pgfpathclose

    \pgfpathcircle{\pgfpoint{\pgf@xc}{\pgf@yc}}{6mm}
  }
}
