% Colours

\definecolor{c4personcolor}{HTML}{08427B}
\definecolor{c4systemcolor}{HTML}{1178bd}
\definecolor{c4existingsystemcolor}{HTML}{999999}
\definecolor{c4existingpersoncolor}{HTML}{999999}
\definecolor{c4containercolor}{HTML}{3c7fc0}
\definecolor{c4componentcolor}{HTML}{85bbf0}

% A C4 node is constructed vertically as follows:
%
% +-----------------
% | head decoration      - horizontal line / head / browser bar...
% | inner ysep
% | nodeparttext
% | inner ysep
% | foot decoration      - just a line
% +---------------
%
%
% north y is at 0.5\ht\nodeparttextebox + inner
% ysep
%
% south y is at 0.5\ht\nodeparttextbox + \dpnodeparttextbox +
% inner yspep
%
%
% when a minimum height is set, this means that center point
% is at minimum height - (headheight + footheight + 2 ysep) / 2
%
% For the basic rectangle shapes, foot and head decorations are just
% lines
%
% The mininum height 

\pgfdeclareshape{c4 rectangle}{
  % Nodes are centered on their label (text part)
  % Text parts are horizontally centered
  \savedanchor\northeast{%
    % Calculate x
    %
    % set x to the width of the text box
    \pgf@x=\the\wd\pgfnodeparttextbox%
    % append margins
    \pgfmathsetlength\pgf@xc{\pgfkeysvalueof{/pgf/inner xsep}}%
    \advance\pgf@x by 2\pgf@xc%
    % check if not smaller than minimum width
    \pgfmathsetlength\pgf@xc{\pgfkeysvalueof{/pgf/minimum width}}%
    \ifdim\pgf@x<\pgf@xc%
    % yes, too small. Enlarge...
    \pgf@x=\pgf@xc%
    \fi%
    %
    \pgf@x=.5\pgf@x%
    % Calculate y
    %
    \pgfmathsetlength\pgf@y{\pgfkeysvalueof{/pgf/inner ysep}}%
    \advance\pgf@y by .5\ht\pgfnodeparttextbox%
  }%

  \savedanchor\southwest{%
    % Calculate x
    %
    % set x to the width of the text box
    \pgf@x=\the\wd\pgfnodeparttextbox%
    % append margins
    \pgfmathsetlength\pgf@xc{\pgfkeysvalueof{/pgf/inner xsep}}%
    \advance\pgf@x by 2\pgf@xc%
    % check if not smaller than minimum width
    \pgfmathsetlength\pgf@xc{\pgfkeysvalueof{/pgf/minimum width}}%
    \ifdim\pgf@x<\pgf@xc%
    % yes, too small. Enlarge...
    \pgf@x=\pgf@xc%
    \fi%
    %
    \pgf@x=-.5\pgf@x%
    % Calculate y
    %
    % First, is total height < minimum height?
    %
    \pgfmathsetlength{\pgf@y}{\pgfkeysvalueof{/pgf/inner ysep}}%
    \pgf@y=2\pgf@y%
    \advance\pgf@y by \ht\pgfnodeparttextbox%
    \advance\pgf@y by \dp\pgfnodeparttextbox%
    %
    \pgfmathsetlength\pgf@yb{\pgfkeysvalueof{/pgf/minimum height}}%
    \ifdim\pgf@y<\pgf@yb%
      % yes, too small. Enlarge...
      \pgf@y=\pgf@yb%
    \fi%
    % south y is full height minus top .. halfway nodeparttextbox
    \pgf@y=-\pgf@y
    \pgfmathsetlength\pgf@yb{\pgfkeysvalueof{/pgf/inner ysep}}%
    \advance\pgf@y by \pgf@yb
    \advance\pgf@y by .5\ht\pgfnodeparttextbox%
  }%
  %
  \inheritanchor[from=rectangle]{center}%
  \inheritanchor[from=rectangle]{north}%
  \inheritanchor[from=rectangle]{north east}%
  \inheritanchor[from=rectangle]{east}%
  \inheritanchor[from=rectangle]{south east}%
  \inheritanchor[from=rectangle]{south}%
  \inheritanchor[from=rectangle]{south west}%
  \inheritanchor[from=rectangle]{west}%
  \inheritanchor[from=rectangle]{north west}%
  %
  \anchor{mid} {%
    \pgf@x=0cm%
    \pgf@y=0cm%
  }%

  \anchor{mid east} {%
    \northeast%
    \pgf@y=0cm%
  }%

  \anchor{mid west} {%
    \southwest%
    \pgf@y=0cm%
  }%
  %
  \anchor{base} {%
    \textpoint%
    \pgf@x=0cm%
  }
  %
  \anchor{base east} {%
    \textpoint%
    \pgf@yb=\pgf@y%
    \northeast%
    \pgf@y=\pgf@yb%
  }%
  %
  \anchor{base west} {%
    \textpoint%
    \pgf@yb=\pgf@y%
    \southwest%
    \pgf@y=\pgf@yb%
  }%
  %
  \inheritanchorborder[from=rectangle]

  % Each of the node parts will be placed on an anchor (not a *saved*
  % anchor!) named after the part

  \savedanchor\textpoint{%
    \pgf@x=-.5\wd\pgfnodeparttextbox%
    \pgf@y=-.5\ht\pgfnodeparttextbox%
  }

  \anchor{text}{\textpoint}%

  % ... and possibly more
  \backgroundpath{% this is new
    % store lower right in xa/ya and upper right in xb/yb
    \southwest \pgf@xa=\pgf@x \pgf@ya=\pgf@y
    \northeast \pgf@xb=\pgf@x \pgf@yb=\pgf@y
    % construct main path
    \pgfpathmoveto{\pgfpoint{\pgf@xa}{\pgf@ya}}
    \pgfpathlineto{\pgfpoint{\pgf@xa}{\pgf@yb}}
    \pgfpathlineto{\pgfpoint{\pgf@xb}{\pgf@yb}}
    \pgfpathlineto{\pgfpoint{\pgf@xb}{\pgf@ya}}
    \pgfpathclose
  }
}

\pgfdeclareshape{c4 person}{
  \inheritsavedanchors[from=c4 rectangle] % this is nearly a rectangle
  \inheritanchorborder[from=c4 rectangle]
  \inheritanchor[from=c4 rectangle]{center}%
  \inheritanchor[from=c4 rectangle]{north}%
  \inheritanchor[from=c4 rectangle]{north east}%
  \inheritanchor[from=c4 rectangle]{east}%
  \inheritanchor[from=c4 rectangle]{south east}%
  \inheritanchor[from=c4 rectangle]{south}%
  \inheritanchor[from=c4 rectangle]{south west}%
  \inheritanchor[from=c4 rectangle]{west}%
  \inheritanchor[from=c4 rectangle]{north west}%
  \inheritanchor[from=c4 rectangle]{mid west}%
  \inheritanchor[from=c4 rectangle]{mid}%
  \inheritanchor[from=c4 rectangle]{mid east}%
  \inheritanchor[from=c4 rectangle]{base}%
  \inheritanchor[from=c4 rectangle]{base west}%
  \inheritanchor[from=c4 rectangle]{base east}%
  \inheritanchor[from=c4 rectangle]{text}%
  \inheritanchorborder[from=c4 rectangle]%

  \backgroundpath{% this is new
    % store lower right in xa/ya and upper right in xb/yb
    \southwest \pgf@xa=\pgf@x \pgf@ya=\pgf@y
    \northeast \pgf@xb=\pgf@x \pgf@yb=\pgf@y
    % construct main path
    \pgfpathmoveto{\pgfpoint{\pgf@xa}{\pgf@ya}}
    \pgfpathlineto{\pgfpoint{\pgf@xa}{\pgf@yb}}
    \pgfpathlineto{\pgfpoint{\pgf@xb}{\pgf@yb}}
    \pgfpathlineto{\pgfpoint{\pgf@xb}{\pgf@ya}}
    \pgfpathclose
  }


}

\tikzset{
c4 system/.style={%
shape=c4 rectangle, draw, fill=c4systemcolor,%
text=white, draw=c4systemcolor!80!black, rounded corners=1ex,%
align=justify, inner sep=1em, text width=13em%
},
c4 existing system/.style={%
shape=c4 rectangle, draw, fill=c4existingsystemcolor,%
text=white, draw=c4existingsystemcolor!80!black, rounded corners=1ex,%
align=justify, inner sep=1em, text width=13em%
},
c4 person/.style={%
shape=c4 person, draw, fill=c4personcolor,%
text=white, draw=c4personcolor!80!black, rounded corners=1ex,%
align=justify, inner sep=1em, text width=13em%
},
c4 existing person/.style={%
shape=c4 person, draw, fill=c4existingpersoncolor,%
text=white, draw=c4existingpersoncolor!80!black, rounded corners=1ex,%
align=justify, inner sep=1em, text width=13em%
}
}
