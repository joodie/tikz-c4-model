% A C4 node contains 3 text parts: text (name), lower,
% and description.
%
% text is the default part of a node, the technology and description
% part are specific to C4 and need special handling below

\newbox\pgfnodeparttechnologybox
\newbox\pgfnodepartdescriptionbox

\pgfdeclareshape{c4 rectangle}{
  \nodeparts{text,technology,description}%

  % Nodes are centered on their label (text part)
  % Text parts are horizontally centered

  \savedanchor\northeast{%
    % Calculate x
    %
    % set x to the width of the widest box
    \pgf@x=\the\wd\pgfnodeparttextbox%
    \pgf@xc=\the\wd\pgfnodeparttechnologybox%
    \ifdim\pgf@x<\pgf@xc%
    \pgf@x=\pgf@xc%
    \fi%
    \pgf@xc=\the\wd\pgfnodepartdescriptionbox%
    \ifdim\pgf@x<\pgf@xc%
    \pgf@x=\pgf@xc%
    \fi%
    % store widest width in @xb
    \pgf@xb=\pgf@x%
    %
    % append margins
    \pgfmathsetlength\pgf@xc{\pgfkeysvalueof{/pgf/inner xsep}}%
    \advance\pgf@x by 2\pgf@xc%
    % check if not smaller than minimum width
    \pgfmathsetlength\pgf@xc{\pgfkeysvalueof{/pgf/minimum width}}%
    \ifdim\pgf@x<\pgf@xc%
    % yes, too small. Enlarge...
    \pgf@x=\pgf@xc%
    \fi%
    %
    % Now, calculate right border: .5 widest box + .5 \pgf@x + outer sep
    \pgf@x=.5\pgf@x%
    \advance\pgf@x by.5\pgf@xb%
    \pgfmathsetlength\pgf@xa{\pgfkeysvalueof{/pgf/outer xsep}}%
    \advance\pgf@x by\pgf@xa%
    % Calculate y
    %
    % TODO: take height of other parts in account here
    % First, is height+depth < minimum height?
    \pgf@y=\ht\pgfnodeparttextbox%
    \advance\pgf@y by\dp\pgfnodeparttextbox%
    \pgfmathsetlength\pgf@yc{\pgfkeysvalueof{/pgf/inner ysep}}%
    \advance\pgf@y by 2\pgf@yc%
    \pgfmathsetlength\pgf@yb{\pgfkeysvalueof{/pgf/minimum height}}%
    \ifdim\pgf@y<\pgf@yb%
    % yes, too small. Enlarge...
    \pgf@y=\pgf@yb%
    \fi%
    % Now, calculate upper border: .5\ht-.5\dp + .5 \pgf@y + outer sep
    \pgf@y=.5\pgf@y%
    \advance\pgf@y by-.5\dp\pgfnodeparttextbox%
    \advance\pgf@y by.5\ht\pgfnodeparttextbox%
    \pgfmathsetlength\pgf@ya{\pgfkeysvalueof{/pgf/outer ysep}}%
    \advance\pgf@y by\pgf@ya%
  }%

  \savedanchor\southwest{%
    % Calculate x
    %
    % set x to the width of the widest box
    \pgf@x=\the\wd\pgfnodeparttextbox%
    \pgf@xc=\the\wd\pgfnodeparttechnologybox%
    \ifdim\pgf@x<\pgf@xc%
    \pgf@x=\pgf@xc%
    \fi%
    \pgf@xc=\the\wd\pgfnodepartdescriptionbox%
    \ifdim\pgf@x<\pgf@xc%
    \pgf@x=\pgf@xc%
    \fi%
    % store widest width in @xb
    \pgf@xb=\pgf@x%
    %
    % append margins
    \pgfmathsetlength\pgf@xc{\pgfkeysvalueof{/pgf/inner xsep}}%
    \advance\pgf@x by 2\pgf@xc%
    % check if not smaller than minimum width
    \pgfmathsetlength\pgf@xc{\pgfkeysvalueof{/pgf/minimum width}}%
    \ifdim\pgf@x<\pgf@xc%
    % yes, too small. Enlarge...
    \pgf@x=\pgf@xc%
    \fi%
    % Now, calculate left border: .5 widest part - .5 \pgf@x - outer sep
    \pgf@x=-.5\pgf@x%
    \advance\pgf@x by.5\pgf@xb
    \pgfmathsetlength\pgf@xa{\pgfkeysvalueof{/pgf/outer xsep}}%
    \advance\pgf@x by-\pgf@xa%
     % Calculate y
    %
    % First, is height+depth < minimum height?
    \pgfmathsetlength{\pgf@y}{\pgfkeysvalueof{/pgf/inner ysep}}%
    \pgf@y=4\pgf@y%
    \advance\pgf@y by \ht\pgfnodepartdescriptionbox%
    \advance\pgf@y by 1.5\pgflinewidth%
    \advance\pgf@y by \ht\pgfnodeparttechnologybox%
    \advance\pgf@y by \dp\pgfnodeparttechnologybox%
    \advance\pgf@y by \dp\pgfnodeparttextbox%
    \advance\pgf@y by \ht\pgfnodeparttextbox%
    \pgfmathsetlength\pgf@yb{\pgfkeysvalueof{/pgf/minimum height}}%
    \ifdim\pgf@y<\pgf@yb%
      % yes, too small. Enlarge...
      \pgf@y=\pgf@yb%
    \fi%
    % Now, calculate upper border: .5\ht-.5\dp - .5 \pgf@y - outer sep
    \pgf@y=-.5\pgf@y%
    \advance\pgf@y by-.5\dp\pgfnodeparttextbox%
    \advance\pgf@y by.5\ht\pgfnodeparttextbox%
    \pgfmathsetlength\pgf@ya{\pgfkeysvalueof{/pgf/outer ysep}}%
    \advance\pgf@y by-\pgf@ya%
  }%

  \inheritanchor[from=rectangle]{center}
  \inheritanchor[from=rectangle]{north}
  \inheritanchor[from=rectangle]{south}
  \inheritanchor[from=rectangle]{west}
  \inheritanchor[from=rectangle]{east}

  % Each of the node parts will be placed on an anchor (not a *saved*
  % anchor!) named after the part

  \savedanchor\technologypoint{%
    \pgf@x=-.5\wd\pgfnodeparttechnologybox%
    \advance\pgf@x by.5\wd\pgfnodeparttextbox%
    \pgfmathsetlength{\pgf@y}{\pgfkeysvalueof{/pgf/inner ysep}}%
    \pgf@y=-2\pgf@y%
    \advance\pgf@y by-\ht\pgfnodeparttechnologybox%
    \advance\pgf@y by-.5\pgflinewidth%
    \advance\pgf@y by-\dp\pgfnodeparttextbox%
    \advance\pgf@y by-.5\pgflinewidth%
  }%
  \anchor{technology}{\technologypoint}%

  \savedanchor\descriptionpoint{%
    \pgf@x=-.5\wd\pgfnodepartdescriptionbox%
    \advance\pgf@x by.5\wd\pgfnodeparttextbox%
    \pgfmathsetlength{\pgf@y}{\pgfkeysvalueof{/pgf/inner ysep}}%
    \pgf@y=-4\pgf@y%
    \advance\pgf@y by-\ht\pgfnodepartdescriptionbox%
    \advance\pgf@y by-.5\pgflinewidth%
    \advance\pgf@y by-\ht\pgfnodeparttechnologybox%
    \advance\pgf@y by-\dp\pgfnodeparttechnologybox%
    \advance\pgf@y by-.5\pgflinewidth%
    \advance\pgf@y by-\dp\pgfnodeparttextbox%
    \advance\pgf@y by-.5\pgflinewidth%
  }%
  \anchor{description}{\descriptionpoint}%

  % ... and possibly more
  \backgroundpath{% this is new
    % store lower right in xa/ya and upper right in xb/yb
    \southwest \pgf@xa=\pgf@x \pgf@ya=\pgf@y
    \northeast \pgf@xb=\pgf@x \pgf@yb=\pgf@y
    % compute corner of ``flipped page''
    \pgf@xc=\pgf@xb \advance\pgf@xc by-5pt % this should be a parameter
    \pgf@yc=\pgf@yb \advance\pgf@yc by-5pt
    % construct main path
    \pgfpathmoveto{\pgfpoint{\pgf@xa}{\pgf@ya}}
    \pgfpathlineto{\pgfpoint{\pgf@xa}{\pgf@yb}}
    \pgfpathlineto{\pgfpoint{\pgf@xb}{\pgf@yb}}
    \pgfpathlineto{\pgfpoint{\pgf@xb}{\pgf@ya}}
    \pgfpathclose
    % add little corner
    \pgfpathmoveto{\pgfpoint{\pgf@xc}{\pgf@yb}}
    \pgfpathlineto{\pgfpoint{\pgf@xc}{\pgf@yc}}
    \pgfpathlineto{\pgfpoint{\pgf@xb}{\pgf@yc}}
    \pgfpathlineto{\pgfpoint{\pgf@xc}{\pgf@yc}}
  }
}

\pgfdeclareshape{c4 person}{
  \inheritsavedanchors[from=rectangle] % this is nearly a rectangle
  \inheritanchorborder[from=rectangle]
  \inheritanchor[from=rectangle]{center}
  \inheritanchor[from=rectangle]{north}
  \inheritanchor[from=rectangle]{south}
  \inheritanchor[from=rectangle]{west}
  \inheritanchor[from=rectangle]{east}
  % ... and possibly more
  \backgroundpath{% this is new
    % store lower left in xa/ya and upper right in xb/yb
    \southwest \pgf@xa=\pgf@x \pgf@ya=\pgf@y
    \northeast \pgf@xb=\pgf@x \pgf@yb=\pgf@y
    % compute corner of ``flipped page''

    \pgf@xc=.5\wd\pgfnodeparttextbox%
    \pgf@yc=\pgf@yb \advance\pgf@yc by 5mm

    \pgfpathmoveto{\pgfpoint{\pgf@xa}{\pgf@ya}}
    \pgfpathlineto{\pgfpoint{\pgf@xa}{\pgf@yb}}
    \pgfpathlineto{\pgfpoint{\pgf@xb}{\pgf@yb}}
    \pgfpathlineto{\pgfpoint{\pgf@xb}{\pgf@ya}}
    \pgfpathclose

    \pgfpathcircle{\pgfpoint{\pgf@xc}{\pgf@yc}}{6mm}
  }
}
